\documentclass[11pt]{article}
\usepackage{times,latexsym,amsmath,amssymb,amsfonts,bm,mathtools}
\usepackage{array, marginnote, paralist, booktabs, multirow, url, natbib}
\usepackage{graphicx, epstopdf, rotating, color, subfig, adjustbox}
\usepackage[linesnumbered,lined,boxed,commentsnumbered]{algorithm2e}
\usepackage[margin=1in]{geometry}
\usepackage[margin=1in]{caption}
\usepackage{relsize}
\usepackage{amsthm}

\SetKwInOut{Parameter}{parameter}
\renewcommand{\familydefault}{ptm}
\setlength{\parindent}{0cm}
\setlength{\parskip}{1em}%
\newcommand{\HRule}{\rule{\linewidth}{0.5mm}}
\newcommand{\mean}{\text{mean}}
\newcommand{\sd}{\text{sd}}
\newcommand{\median}{\text{median}}
\newcommand{\IQR}{\text{IQR}}
\newcommand{\MAD}{\text{MAD}}
\newcommand{\dist}{\text{dist}}
\newcommand{\diag}{\text{diag}}
\newcommand{\nn}{\text{nn}}
\newcommand{\nnd}{\text{nnd}}
%\DeclareMathOperator*{\argmax}{arg\,max}
%\DeclareMathOperator*{\argmin}{arg\,min}
\newcommand{\argmin}{\mathop{\text{argmin}}}
\newcommand{\argmax}{\mathop{\text{argmax}}}
\newcolumntype{d}[1]{D{.}{.}{#1}}  % define "d" column type

\newcommand{\density}{\text{density}}
\newcommand{\AUC}{\text{AUC}}
\newcommand{\margincomment}[1]{\marginpar{\footnotesize{#1}}}
\newtheorem{theorem}{Theorem}[section]
\newtheorem{definition}[theorem]{Definition}
\newtheorem{proposition}[theorem]{Proposition}

\graphicspath{{./Graphics/}}

\newcommand{\chk}{$\checkmark$}
\newcolumntype{R}[2]{%
    >{\adjustbox{angle=#1,lap=\width-(#2)}\bgroup}%
    l%
    <{\egroup}%
}
\newcommand*\rot{\multicolumn{1}{R{90}{1em}}}% no optional argument here, please!

% =======================================================================
\begin{document}
% =======================================================================
\title{Basis vectors suitable for outlier detection}
\author{Sevvandi Kandanaarachchi, Rob J. Hyndman}
\maketitle
\abstract{FOR LATER}

% =======================================================================
\section{Introduction}
% =======================================================================
FOR LATER!

% =======================================================================
\section{Related work}
% =======================================================================
FOR LATER!

% =======================================================================
\section{Mathematical Framework}\label{sec:MathFrame}
% =======================================================================
Let $X_{N \times p}$ be a matrix denoting a dataset of $N$ observations and $p$ attributes. Let us denote the  $i^{\text{th}}$ observation in $X$ by $\bm{x}_i$. The  distance between points $\bm{x}_i$ and $\bm{x}_j$  can be written as 
\begin{equation}\label{eq:secMF1}
\dist(\bm{x}_i, \bm{x}_j)^2 = \left( \bm{x}_i - \bm{x}_j \right)^T S \left( \bm{x}_i - \bm{x}_j \right) \, , 
\end{equation}
where $S$ is a symmetric positive definite matrix. In addition when $S$ is diagonal, i,e, $S = \diag(s_1, s_2, \ldots s_p)$ we get
\begin{equation}\label{eq:secMF2}
    \dist(\bm{x}_i, \bm{x}_j)^2 = \left\langle \eta\, ,  \left( \bm{x}_i - \bm{x}_j \right)^2 \right\rangle\, 
\end{equation}
where $\left\langle \cdot\, , \cdot \right\rangle$ denotes the standard inner product in $\mathbb{R}^p$,   $\eta = \left(s_1, s_2, \ldots s_p\right)^T$, and with some abuse of notation $\left( \bm{x}_i - \bm{x}_j \right)^2$ denotes the element wise difference of $\left( \bm{x}_i - \bm{x}_j \right)$ squared.
% Noting that normalization is a standard pre-processing step in data analysis, we denote the associated the normalized observation by $\tilde{\bm{x}}_i$. Thus, we can write
% \begin{equation}\label{eq:secMF1}
%     \tilde{\bm{x}}_i = S^{-1}\left(\bm{x} - \mu\right) \, ,
% \end{equation}
% where $\mu$ is generally the column-wise mean or minimum of $X$ and $S$ is the matrix containing column-wise standard deviations or ranges in its diagonal.

% The distance between two points $\tilde{\bm{x}}_i$ and $\tilde{\bm{x}}_j$ is 



% =======================================================================
\section{Experiments}
% =======================================================================
FOR LATER!

% =======================================================================
\section{Results with visualization}
% =======================================================================
FOR LATER!

% =======================================================================
\section{Results on a data repository}
% =======================================================================
FOR LATER!

% =======================================================================
\section{Conclusion}
% =======================================================================
FOR LATER!
\end{document}
